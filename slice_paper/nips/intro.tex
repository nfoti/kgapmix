\section{Introduction} 

Nonparametric mixture models allow us to bypass the
issue of model selection, by modeling data using a random number of mixture components
that can grow if we observe more data. However, such models work on the
assumption that data can be considered exchangeable. This assumption often does
not hold in practice as distributions commonly vary with some covariate. For
example, the proportions of different species may vary across geographic
regions, and the distribution over topics discussed on Twitter is likely to 
evolve over time.

Recently, there has been increasing interest in \emph{dependent} nonparametric
processes \cite{MacEachern:1999}, that extend existing nonparametric
distributions to non-stationary data. While a nonparametric process is a
distribution over a single measure, a dependent nonparametric process is a
distribution over a \emph{collection} of measures, which may be associated with
values in a covariate space. The key property of a dependent nonparametric
process is that the measure at each covariate value is marginally distributed
according to a known nonparametric process.

A number of dependent nonparametric processes have been developed in the
literature (\cite{Dunson:2010} $\S6$). For example, the single-p DDP \cite{MacEachern:1999} defines a
collection of Dirichlet processes with common atom sizes but variable atom
locations. The order-based DDP \cite{Griffin:Steel:2006} constructs a collection of
Dirichlet processes using a common set of beta random variables, but permuting
the order in which they are used in a stick-breaking construction. The Spatial
Normalized Gamma Process (SNGP) \cite{Rao:Teh:2009} defines a gamma process on
an augmented space, such that at each covariate location a subset of the atoms
are available. This creates a dependent gamma process, that can be normalized
to obtain a dependent Dirichlet process. The kernel beta process (KBP)
\cite{Ren:Wang:Dunson:Carin:2011}
defines a beta process on an augmented space, and at each covariate location
modulates the atom sizes using a collection of kernels, to create a collection 
of dependent beta processes.

Unfortunately, while such models have a number of appealing properties,
inference can be challenging. While there are many similarities between
existing dependent nonparametric processes, most of the inference schemes that
have been proposed are highly specific, and cannot be generally applied without
significant modification. 

The contributions of this paper are twofold.  First, in Section~\ref{sec:dep}
we describe a general class of dependent nonparametric processes, based on
defining completely random measures on an extended space.  This class of models
includes the SNGP and the KBP as special cases. Second, we develop a
slice sampler that is applicable for all the dependent probability measures in
this framework. We compare our slice sampler to existing inference algorithms,
and show that we are able to achieve superior performance over existing
algorithms. Further, the generality of our algorithm mean we are able to easily
modify the assumptions of existing models to better fit the data, without the
need to significantly modify our sampler.

%\begin{itemize}
%\item Why dependent nonparametric models are useful.
%\item Briefly mention existing DDPs and CRMs
%\item Inference difficulties
%\item Summary of contributions:
%  \begin{itemize}
%  \item Define a class of dependent normalized random measures that includes the SNGP as a special case.
%  \item Develop a slice sampling algorithm that can be generally applied and has good performance.
%\end{itemize}
%\end{itemize}

